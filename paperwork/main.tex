\documentclass[a4paper, 12pt]{article}
\usepackage[a4paper]{geometry}

\usepackage[T1,T2A]{fontenc}
\usepackage[utf8]{inputenc}
\usepackage[english,russian]{babel}
\usepackage{libertine}

\usepackage{amsmath}
\usepackage{amssymb}
\usepackage{amsthm}
\usepackage{mathrsfs}
\usepackage{mathtools}

\title{Нильпотентные и разрешимые алгебры Ли}
\author{Виногродский Серафим}
\date{\today}

\newtheorem{theorem}{Теорема}[section]
\newtheorem{corollary}{Следствие}[theorem]
\newtheorem{lemma}[theorem]{Лемма}
\newtheorem{proposition}[theorem]{Утверждение}

\theoremstyle{definition}
\newtheorem{definition}{Определение}[section]

\begin{document}
\maketitle
\tableofcontents
\thispagestyle{empty}
\pagebreak

\section{Введение}%
\label{sec:introduction}

\subsection{Основные понятия}%
\label{sub:lie_algebra_notion}

\begin{definition}
    \label{def:lie_algebra}
    Векторное пространство \( L \) над полем \( \mathrm F \), дополненное операцией \( L \times L \to L \), которая обозначается \( (x ,y) \mapsto [x, y] \) и называется \textit{скобкой Ли} или \textit{коммутатором} \( x \) и \( y \), называется \textit{алгеброй Ли} над полем \( \mathrm F \), если выполнен следующий ряд аксиом:
\begin{itemize}
    \item[(\( L 1 \))] Скобка Ли билинейна.
    \item[(\( L 2 \))] \( [x, x] = 0 \) для любого \( x \in L \).
    \item[(\( L 3 \))] \( [x, [y, z]] + [y, [z, x]] + [z, [x, y]] = 0 \quad (x, y, z \in L) \).
\end{itemize}
\end{definition}

Аксиома (\( L 3 \)) называется \textit{тождеством Якоби}.
Из аксиом (\( L 1 \)) и (\( L 2 \)), применённых к скобке \( [x + y, x + y] \), следует антикоммутативность скобки Ли:
\begin{itemize}
    \item[(\( L 2' \))] \( [x, y] = -[y, x] \) для любых \( x, y \in L \).
\end{itemize}
Обратно, если \( \operatorname{char} \mathrm F \neq 2 \), то из утверждение (\( L 2' \)) тривиально следует из аксиомы (\( L 2 \)) и потому для таких полей (\( L 2 \)) эквивалентна (\( L 2' \)).

\begin{definition}
    \label{def:isomorphous_algebras}
    Две алгебры Ли \( L \), \( L' \) называются \textit{изоморфными}, если существует такой изоморфизм векторных пространств \( \phi : L \to L' \), что
    \[
        \phi([x, y]) = [\phi(x), \phi(y)] \quad \forall x, y \in L.
    \]
    Само отображение \( \phi \) при этом называется \textit{изоморфизмом} алгебр Ли.
\end{definition}

\begin{definition}
    \label{def:subalgebras}
    Подпространство \( K \) алгебры Ли \( L \) называется \textit{подалгеброй} алгебры \( L \), если \( K \) замкнуто относительно скобки Ли, т.е.
    \[
        \forall x, y \in K \quad [x, y] \in K.
    \]
\end{definition}

Нетрудно показать, что само подпространство \( K \) вместе с индуцированными операциями также является алгеброй Ли.

Любая алгебра Ли \( L \) имеет как минимум две тривиальные (\textit{несобственные}) подалгебры отвечающие тривиальным подпространствам: \( \left\{ 0 \right\} \) и \( L \).
Также, если \( L \neq \left\{ 0 \right\} \), то любой ненулевой элемент \( v \in L \) задаёт одномерную подалгебру \( Fv \). Умножение в такой алгебре тривиально, поскольку в силу аксиом (\( L 1 \)) и (\( L 2 \)) имеем \( [x, y] = 0 \) для любых \( x, y \in Fv \).

\subsection{Линейные алгебры Ли}
\label{sub:linear_li_algebras}

Пусть \( V \) --- конечномерное векторное пространство над полем \( \mathrm F \). Обозначим через \( \operatorname{End} V \) множество всех эндоморфизмов в пространстве \( V \). Тогда \( \operatorname{End} V \) --- векторное пространство размерности \( n^2 \) (где \( n = \dim V \)) над полем \( \mathrm F \) и одновременно \( \operatorname{End} V \) --- кольцо относительно операции умножения операторов. Определим новую операцию \( [x, y] = xy - yx \), называемую \textit{скобкой} или \textit{коммутатором} элементов \( x \) и \( y \). Вместе с ней \( \operatorname{End} V \) становится алгеброй Ли над полем \( \mathrm F \): выполнение аксиом (\( L 1 \)) и (\( L 2 \)) очевидно, а аксиома (\( L 3 \)) напрямую следует из (\( L 1 \)) и (\( L 2 \)). Чтобы отличать полученную алгебраическую структуру от изначальной ассоциативной структуры кольца, мы будем обозначать \( \operatorname{End} V \) как \( \mathfrak{gl}(V) \), когда она рассматривается как алгебра Ли.

\begin{definition}
    Алгебра \( \mathfrak{gl}(V) \) называется \textit{полной линейной алгеброй}.
\end{definition}

\begin{definition}
    Любая подалгебра \( \mathfrak{gl}(V) \) называется \textit{линейной алгеброй}.
\end{definition}

Зафиксировав базис в пространстве \( V \), можно отождествить \( \mathfrak{gl}(V) \) с множеством всех матриц размера \( n \times n \) над полем \( \mathrm F \), обозначаемым \( \mathfrak{gl}(n, \mathrm F) \), что удобно при выполнении вычислений в явном виде. Для дальнейших ссылок приведём здесь таблицу коммутирования для \({ \mathfrak{gl}(n, \mathrm F) }\) в стандартном базисе, состоящем из матриц \({ e_{ij} }\) (у которых в позиции \({ (i, j) }\) стоит \({ 1 }\), а в остальных \({ 0 }\)). Поскольку \({ e_{ij} e_{kl} = \delta_{jk}  e_{il} }\) (где \({ \delta_{jk} \in \left\{ 0, 1 \right\} }\)), мы получаем, что
\[
    [e_{ij}, e_{kl}] = \delta_{jk} e_{il} - \delta_{li} e_{kj}.
\]

Рассмотрим теперь некоторые примеры линейных алгебр, играющих основную роль в этой работе наряду с \( \mathfrak{gl}(V) \). Они разделяются на четыре семейства: \( \mathbf{A}_l, \mathbf{B}_l, \mathbf{C}_l, \mathbf{D}_l \) (где \( l \geqslant 1 \)) --- и называются \textit{классическими алгебрами}. В примерах \( \mathbf{B}_l \) ---  \( \mathbf{D}_l \) будем считать, что \( \operatorname{char} \mathrm F \neq 2 \).

\paragraph{\( \mathbf{A}_l \):}
Пусть \( \dim V = l + 1 \). Обозначим через \( \mathfrak{sl}(V) \) или \( \mathfrak{sl}(l + 1, \mathrm F) \) множество всех эндоморфизмов в пространстве \( V \), имеющих нулевой след. Поскольку
\begin{align*}
    \operatorname{Tr}(xy) &= \operatorname{Tr}(yx), \\
    \operatorname{Tr}(x + y) &= \operatorname{Tr}(x) + \operatorname{Tr}(y),
\end{align*}
множество \( \mathfrak{sl}(V) \) замкнуто относительно коммутирования и потому является подалгеброй \( \mathfrak{gl}(V) \), называемой \textit{специальной линейной алгеброй}.

Найдём теперь размерность \( \mathfrak{sl}(V) \). С~одной стороны \( \mathfrak{sl}(V) \) --- собственная подалгебра \( \mathfrak{gl}(V) \), так что её размерность не может быть больше \({ (l + 1)^2 - 1 }\). С~другой стороны нетрудно явно предоставить такое количество линейно независимых матриц с нулевым следом:
\[
    \left\{ e_{ij} \mid i \neq j \right\} \cup \left\{ e_{ii} - e_{i + 1, i + 1} \mid 1 \leqslant i \leqslant l \right\}.
\]
Этот базис будем считать стандартным в пространстве \({ \mathfrak{sl}(l + 1, \mathrm F) }\).

\paragraph{\( \mathbf{C}_l \):} \ldots
\paragraph{\( \mathbf{B}_l \):} \ldots
\paragraph{\( \mathbf{D}_l \):} \ldots \\

Отметим также несколько примеров, играющих далее вспомогательную роль. Пусть \( \mathfrak{t}(n, \mathrm F) \) --- множество всех верхнетреугольных матриц, \( \mathfrak{n}(n, \mathrm F) \) --- множество строго верхнетреугольных матриц и \( \mathfrak{d}(n, \mathrm F) \) --- множество всех диагональных матриц. Тривиально проверяется, что каждое из этих множеств замкнуто относительно коммутирования.

\subsection{Абстрактные алгебры Ли}
Мы рассмотрели определённое количество естественных примеров линейных алгебр Ли. Иногда, однако, бывает полезно рассматривать и абстрактные алгебры Ли. Например, любое векторное пространство \({ L }\) над полем \({ \mathrm F }\) можно превратить в алгебру Ли с тривиальным умножением, задав \({ [x, y] = 0 }\) для любых \({ x, y \in L }\). Такая алгебра называется \textit{абелевой} (поскольку в линейном случае равенство \({ [x, y] = 0 }\) означает, что \({ x }\) и \({ y }\) коммутируют.)

Из билинейности скобки Ли следует, что если \({ L }\) --- алгебра Ли с базисом \({ x_1, \ldots, x_n }\), то всю её таблицу умножения можно восстановить по структурным константам \({ a^{k}_{ij} }\), которые входят в выражения
\[
    [x_i, x_j] = \sum_{k=1}^{n} a^{k}_{ij} x_k.
\]
Более того, константы \({ a^{k}_{ij} }\), для которых \({ i \geqslant j }\), восстанавливаются по остальным в силу свойств (\({ L 2 }\)) и (\({ L 2' }\)). Обратно, можно с нуля определить абстрактную алгебру Ли, задав семейство структурных констант \({ \left\{ a^{k}_{ij} \right\} }\). Естественно, подойдёт не всякое такое семейство. Чтобы заданная таким образом операция коммутирования удовлетворяла аксиомам (\({ L 2 }\)) и (\({ L 3 }\)), должны выполняться следующие соотношения:
\[
    \begin{gathered}
        a^{k}_{ii} = a^{k}_{ij} + a^{k}_{ji} = 0; \\
        \sum_{k=1}^{n} \left( a^{k}_{ij} a^{m}_{kl} + a^{k}_{jl} a^{m}_{ki} + a^{k}_{li} a^{m}_{kj} \right) = 0.
    \end{gathered}
\]

\section{Идеалы и гомоморфизмы}
\subsection{Идеалы}

\begin{definition}
    Подпространство \({ I }\) алгебры Ли \({ L }\) называется \textit{идеалом} в \({ L }\), если для любых \({ x \in L,\: y \in I }\) имеем \({ [x, y] \in I }\). (Поскольку \({ [x, y] = -[y, x] }\), это условие можно записать и как \({ [y, x] \in I }\).)
\end{definition}

Очевидно, что любая алгебра Ли \({ L }\) имеет два тривиальных (\textit{собственных}) идеала: \({ \left\{ 0 \right\} }\) и \({ L }\). Менее тривиальный пример --- так называемый \textit{центр}
\[
    Z(L) \overset{\text{def}}= \left\{ z \in L \mid [x, z] = 0 \quad \forall x \in L \right\}.
\]

\begin{definition}
    Подалгебра всех линейных комбинаций коммутаторов произвольных элементов алгебры Ли \({ L }\) обозначается \({ [L, L] }\) и называется \textit{производной алгеброй} алгебры \({ L }\).
\end{definition}

Очевидно, что \({ [L, L] }\) является идеалом алгебры \({ L }\). Так же ясно, что алгебра \({ L }\) является абелевой тогда и только тогда, когда \({ [L, L] = \left\{ 0 \right\} }\).

Если \({ I, J }\) --- идеалы в \({ L }\), то и \({ I + J }\) --- тоже идеал в \({ L }\). Аналогично идеалом является и \({ [I, J] }\), где
\[
    [I, J] \overset{\text{def}}= \left\{ \sum_i [x_i, y_i] : \left\{ x_i \right\} \subset I, \left\{ y_i \right\} \subset J \right\}.
\]
Производная алгебра \({ [L, L] }\) --- частный случай этой конструкции.

\begin{definition}
    Если в алгебре Ли \({ L }\) нет идеалов, кроме самой \({ L }\) и \({ \left\{ 0 \right\} }\), и при этом \({ [L, L] \neq \left\{ 0 \right\} }\) (т.е. \({ L }\) не является абелевой), то алгебра \({ L }\) называется \textit{простой}.
\end{definition}

Условие \({ [L, L] \neq \left\{ 0 \right\} }\) накладывается для того, что бы не придавать излишнего значения одномерным алгебрам. Нетрудно показать, что для любой простой алгебры \({ L }\) всегда имеем \({ Z(L) = \left\{ 0 \right\} }\) и \({ [L, L] = L }\).

\paragraph{Пример.} Пусть \({ L = \mathfrak{sl}(2, \mathbb R) }\). Выберем тогда стандартный базис в \({ L }\), состоящий из трёх матриц (см. параграф (\ref{sub:linear_li_algebras})):
\[
    x = \begin{bmatrix}
        0 & 1 \\
        0 & 0
    \end{bmatrix}, \quad
    y = \begin{bmatrix}
        0 & 0 \\
        1 & 0
    \end{bmatrix}, \quad
    h = \begin{bmatrix}
        1 & 0 \\
        0 & -1
    \end{bmatrix}.
\]
Тогда таблица коммутирования в алгебре \({ L }\) полностью определяется следующими соотношениями:
\[
    \begin{gathered}
        [x, y] = h, \quad  [h, x] = 2x, \quad [h, y] = -2y.
    \end{gathered}
\]
Пусть \({ I }\) --- ненулевой идеал в алгебре \({ L }\) и \({ ax + by + ch }\) --- некоторый ненулевой элемент в \({ I }\). Дважды применяя к нему оператор \({ \operatorname{ad} x }\), получаем \({ -2b x \in I }\), а дважды применяя оператор \({ \operatorname{ad} y }\), получаем \({ -2ay \in I }\). Поэтому, если \({ a }\) или \({ b }\) отлично от нуля, то \({ I }\) содержит \({ y }\) или \({ x }\), но тогда из определения идеала следует, что \({ x, y, h \in I }\), а значит \({ I = L }\). С другой стороны, если \({ a = b = 0 }\), то \({ 0 \neq ch \in I }\), что аналогичным образом влечёт \({ I = L }\). Получаем, что \({ L }\) --- \textit{простая алгебра}.

\begin{definition}
    Пусть \({ L }\) --- алгебра Ли, \({ I }\) --- собственный идеал в \({ L }\). Тогда факторпространство \({ L / I }\) с определённой на нём скобкой Ли:
    \[
        [x + I, y + I] \overset{\text{def}}= [x, y] + I,
    \]
    называется \textit{факторалгеброй} \({ L }\) по идеалу \({ I }\) и так же обозначается \({ L / I }\).
\end{definition}

Определение скобки Ли в пространстве \({ L / I }\) корректно и не зависит от выбора представлений \({ x }\) и \({ y }\) для классов эквивалентности. Действительно, если \({ x' = x + I }\), \({ y' = y + I }\), то имеем \({ x' = x + u }\), \({ y' = y + v }\) (где \({ u, v \in I }\)), откуда по линейности скобки Ли
\[
    [x', y'] = [x, y] + \underbrace{([u, y] + [x, v] + [u, v])}_{\text{элемент из \({ I }\)}},
\]
а значит \({ [x', y'] + I = [x, y] + I }\).

\subsection{Гомоморфизмы}
\begin{definition}
    Пусть \({ L, L' }\) --- две линейные алгебры над полем \({ F }\). Линейное отображение \({ \phi : L \to L' }\) называется \textit{гомоморфизмом} алгебр Ли, если
    \[
        \phi([x, y]) = [\phi(x), \phi(y)] \quad \forall x, y \in L.
    \]
\end{definition}

\begin{definition}
    Гомоморфизм алгебр Ли \({ \phi : L \to L' }\) называется
    \begin{itemize}
        \item \textit{мономорфизмом}, если \({ \ker \phi = \left\{ 0 \right\} }\);
        \item \textit{эпиморфизмом}, если \({ \operatorname{im} \phi = L' }\).
    \end{itemize}
\end{definition}

Очевидно, что \({ \phi }\) является изоморфизмом тогда и только тогда, когда \({ \phi }\) одновременно и моно- и эпиморфизм. Легко так же проверяется, что \({ \ker \phi }\) --- идеал в \({ L }\), а \({ \operatorname{im} \phi }\) --- подалгебра алгебры \({ L }\).

Как и в других алгебраических теориях, для алгебр Ли существует естественное взаимно однозначное соответствие между гомоморфизмами и идеалами: гомоморфизму \({ \phi }\) ставится в соответствие идеал \({ \ker \phi }\), а идеалу \({ I }\) --- \textit{каноническое отображение} \({ \pi : L \to L / I }\), заданное правилом \({ x \mapsto x + I }\).

Для изоморфизмомов алгебр Ли выполняются и классические теоремы об изоморфизме:

\begin{proposition}
    \label{prop:1st_isomorphism_theorem}
    Если \({ \phi : L \to L }\) --- гомоморфизм алгебр Ли, то \( L / \ker \phi \simeq \operatorname{im} \phi. \)
\end{proposition}
\begin{proof}
    Легко показать, что отображение \({ (x + \ker \phi) \mapsto \phi(x) }\) есть изоморфизм алгебр \({ L / \ker \phi }\) и \({ \operatorname{im} \phi }\).
\end{proof}

\begin{proposition}
    Если \({ I }\) и \({ J }\) --- два идеала в алгебре \({ L }\) и \({ I \subset J }\), то \({ J / I }\) --- идеал в \({ L / I }\), а алгебра \({ (L / I) / (J / I) }\) изоморфна \({ L / J }\).
\end{proposition}
\begin{proof}
    Пусть \({ x + I \in J / I }\) и \({ y + I \in L / I }\). Тогда имеем
    \[
        [x + I, y + I] = [x, y] + I \in J / I,
    \]
    а значит \({ J / I }\) --- идеал в \({ L / I }\). Рассмотрим теперь гомоморфизм \({ \phi : x + I \mapsto x + J }\), действующий из \({ L / I }\) в \({ L / J }\). Отображение \({ \phi }\) задано корректно, поскольку если \({ x' + I = x + I }\), то \({ x' = x + i }\) (где \({ i \in I \subset J }\)), а значит
    \[
        \phi(x' + I) = (x + i) + J = x + J.
    \]
    Очевидно, что \({ \ker \phi = J / I }\), \({ \operatorname{im} \phi = L / I }\), но тогда по утверждению (\ref{prop:1st_isomorphism_theorem}) имеем \( (L / I) / (J / I) \simeq L / J \).
\end{proof}

\begin{proposition}
    Если \({ I }\) и \({ J }\) --- два идеала в алгебре \({ L }\), то \({ (I + J) / J \simeq I / (I \cap J) }\).
\end{proposition}
\begin{proof}
    Рассмотрим гомоморфизм \({ \phi : i \mapsto i + J }\), действующий из \({ I }\) в \({ (I + J) / J }\). Очевидно, что \({ \ker \phi = I \cap J }\), \({ \operatorname{im} \phi = (I + J) / J }\), но тогда по утверждению (\ref{prop:1st_isomorphism_theorem}) имеем \({ I / (I \cap J) \simeq (I + J) / J }\).
\end{proof}

\section{Разрешимые и нильпотентные алгебры Ли}
\subsection{Разрешимость}

Определим прежде всего следующую последовательность идеалов алгебры Ли \({ L }\) (\textit{производный ряд}):
\[
    L^{(0)} = L, \quad L^{(k)} = [L^{(k - 1)}, L^{(k - 1)}] \quad (k > 0).
\]

\begin{definition}
Алгебра Ли \({ L }\) называется \textit{разрешимой}, если \({ L^{(n)} = \left\{ 0 \right\} }\) при некотором \({ n }\). В противном случае алгебра \({ L }\) называется \textit{неразрешимой}.
\end{definition}

В частности, абелевы алгебры всегда разрешимы, а простые алгебры заведомо неразрешимы.

Достаточно общим примером разрешимой алгебры Ли является алгебра \({ L = \mathfrak{t} (n, F) }\) верхнетреугольных матриц, введённая в параграфе (\ref{sub:linear_li_algebras}). Базис в \({ L }\) состоит из матричных единиц \({ e_{ij} }\) для которых \({ i \leqslant j }\). Размерность \({ L }\), как следствие, равна \({ 1 + 2 + \cdots + n = n(n + 1)/2 }\). Чтобы показать, что алгебра \({ L }\) разрешима, вычислим в явном виде её производный ряд, используя формулу для коммутаторов из параграфа (\ref{sub:linear_li_algebras}). В первую очередь, имеем \({ [e_{ii}, e_{il}] = e_{il} }\) для \({ i < l }\), откуда следует, что \({ \mathfrak{n}(n, \mathrm F) \subset [L, L] }\), где \({ \mathfrak{n}(n, \mathrm F) }\) --- подалгебра строго верхнетреугольных матриц. Поскольку \({ \mathfrak{t}(n ,\mathrm F) = \mathfrak{d}(n, \mathrm F) + \mathfrak{n}(n, \mathrm F) }\) и \({ \mathfrak{d}(n, \mathrm F) }\) абелева, можно заключить, что \({ L^{(1)} = [L, L] = \mathfrak{n}(n, \mathrm F) }\).

Работая далее в алгебре \({ \mathfrak{n}(n, \mathrm F) }\) естественно определить <<уровень>> матрицы \({ e_{ij} }\), как число \({ j - i }\). Рассмотрим теперь произвольный коммутатор \({ [e_{ij}, e_{kl}] }\) базисных элементов в \({ \mathfrak{n}(n, \mathrm F) }\). Имеем \({ i < j, k < l }\); без ограничения общности можно так же полагать, что \({ i \neq l }\) (поскольку в обратном случае \({ [e_{ij}, e_{ki}] = -[e_{ki}, e_{ij}] }\) и, если здесь \({ k = j }\), то коммутатор равен \({ 0 }\) и не представляет интереса). Тогда имеем
\[
    [e_{ij}, e_{kl}] = \begin{cases}
        e_{il}, & j = k, \\
        0, & j \neq k.
    \end{cases}
\]
В частности любой элемент \({ e_{il} \in L^{(2)} }\) есть коммутатор двух базисных матриц, уровни которых в сумме дают уровень \({ e_{il} }\), а значит \({ L^{(2)} }\) есть линейная оболочка элементов \({ e_{ij} }\), уровень которых больше либо равен \({ 2 }\). Аналогично каждая последующая производная алгебра \({ L^{(m)} }\) есть линейная оболочка элементов \({ e_{ij} }\), уровень которых больше либо равен \({ 2^{m - 1} }\). Наконец, очевидно, что \({ L^{(m)} = \left\{ 0 \right\} }\), когда \({ 2^{m - 1} > n - 1 }\).

Приведём теперь несколько простых свойств разрешимых алгебр Ли.

\begin{proposition}
    Если алгебра Ли \({ L }\) разрешима, то разрешимы и все её подалгебры.
\end{proposition}
\begin{proof}
    Если \({ K }\) --- подалгебра в \({ L }\), то из определения (\ref{def:subalgebras}) имеем \({ K^{(i)} \subset L^{(i)} }\). Тогда если для некоторого \({ k }\) имеем \({ L^{(k)} = \left\{ 0 \right\} }\), то и \({ K^{(k)} = \left\{ 0 \right\} }\).
\end{proof}

\begin{proposition}
    Если алгебра Ли \({ L }\) разрешима, то разрешимы и все её гомоморфные образы.
\end{proposition}
\begin{proof}
    TODO (надо сначала ввести определение гомоморфизма)
\end{proof}

\begin{proposition}
    Если \({ I }\) --- разрешимый идеал в алгебре Ли \({ L }\) и алгебра \({ L / I }\) разрешима, то и сама \({ L }\) тоже разрешима.
\end{proposition}
\begin{proof}
    TODO (надо сначала ввести определение алгебры \({ L / I }\))
\end{proof}

\begin{proposition}
    Если \({ I }\) и \({ J }\) --- разрешимые идеалы в алгебре Ли \({ L }\), то идеал \({ I + J }\) тоже разрешим.
\end{proposition}
\begin{proof}
    TODO (тоже нужны гомоморфизмы)
\end{proof}

\end{document}

% vim: spell spelllang=ru_yo
