\documentclass[a4paper, 12pt]{article}
\usepackage[a4paper]{geometry}

\usepackage[T1,T2A]{fontenc}
\usepackage[utf8]{inputenc}
\usepackage[english,russian]{babel}
\usepackage{libertine}

\usepackage{amsmath}
\usepackage{amssymb}
\usepackage{amsthm}
\usepackage{mathrsfs}
\usepackage{mathtools}

\title{Нильпотентные и разрешимые алгебры Ли}
\author{Виногродский Серафим}
\date{\today}

\newtheorem{theorem}{Теорема}[section]
\newtheorem{corollary}{Следствие}[theorem]
\newtheorem{lemma}[theorem]{Лемма}

\theoremstyle{definition}
\newtheorem{definition}{Определение}[section]

\begin{document}
\maketitle
\tableofcontents
\thispagestyle{empty}
\pagebreak

\section{Введение}%
\label{sec:introduction}

\subsection{Основные понятия}%
\label{sec:definitions}

Рассмотрим для начала определение алгебры Ли, основные связанные с ней понятия.

\begin{definition}
    \label{def:lie_algebra}
    Векторное пространство \( L \) над полем \( \mathrm F \), дополненное операцией \( L \times L \to L \), которая обозначается \( (x ,y) \mapsto [xy] \) и называется скобкой Ли или коммутатором \( x \) и \( y \), называется алгеброй Ли над полем \( \mathrm F \), если выполнен следующий ряд аксиом:
    \begin{itemize}
        \item[(\textit{L}1)] Скобка Ли билинейна.
        \item[(\textit{L}2)] \( [x x] = 0 \) для любого \( x \in L \).
        \item[(\textit{L}3)] Для скобки Ли выполнено тождество Якоби, т.е.
            \[
                [x[yz]] + [y[zx]] + [z[xy]] = 0 \quad (x, y, z \in L).
            \]
    \end{itemize}
\end{definition}

\begin{theorem}%
    \label{theorem:commutator_anticommutativity}
    Операция коммутировая антикоммутативна, т.е.
    \[
        [x y] = -[y x] \quad \forall x, y \in L.
    \]
\end{theorem}
\begin{proof}
    Рассмотрим два произвольных \( x, y \in L \). Тогда по аксиоме (\textit{L}2) имеем \( [x + y, x + y] = 0 \) и одновременно по аксиоме (\textit{L}1)
    \begin{align*}
        [x + y, x + y]
        &= [x x] + [x y] + [y x] + [y y] \\
        &= [x y] + [y x].
    \end{align*}

    Получаем, что \( [x y] + [y x] = 0 \), откуда и следует, что \( [x y] = -[y x] \).
\end{proof}

\begin{definition}
    \label{def:isomorphism}
    Изоморфизмом двух алгебр Ли \( L \), \( L' \) называется такой изоморфизм векторных пространств \( \phi : L \to L' \), что
    \[
        \phi([x y]) = [\phi(x) \phi(y)] \quad \forall x, y \in L.
    \]
\end{definition}

\begin{definition}
    \label{def:isomorphousness}
    Две алгебры Ли \( L \), \( L' \) называются изоморфными, если существует изоморфизм алгебр Ли \( \phi : L \to L' \).
\end{definition}

\begin{definition}
    Подпространство \( K \) алгебры Ли \( L \) называется подалгеброй алгебры \( L \), если \( \forall x, y \in K \quad [x y] \in K \).
\end{definition}

Нетрудно показать, что \( K \) вместе с наследованными операциями удовлетворяет всем аксиомам из определения алгебры Ли.

\end{document}

% vim: spell spelllang=ru_yo
