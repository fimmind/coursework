\documentclass[a4paper, 12pt]{article}
\usepackage[a4paper]{geometry}

\usepackage[T1,T2A]{fontenc}
\usepackage[utf8]{inputenc}
\usepackage[english,russian]{babel}
\usepackage{libertine}

\usepackage{amsmath}
\usepackage{amssymb}
\usepackage{amsthm}
\usepackage{mathrsfs}
\usepackage{mathtools}

\title{Нильпотентные и разрешимые алгебры Ли}
\author{Виногродский Серафим}
\date{\today}

\newtheorem{theorem}{Теорема}[section]
\newtheorem{corollary}{Следствие}[theorem]
\newtheorem{lemma}[theorem]{Лемма}

\theoremstyle{definition}
\newtheorem{definition}{Определение}[section]

\begin{document}
\maketitle
\tableofcontents
\thispagestyle{empty}
\pagebreak

\section{Введение}%
\label{sec:introduction}

\subsection{Основные понятия}%
\label{sec:lie_algebra_notion}

\begin{definition}
    \label{def:lie_algebra}
    Векторное пространство \( L \) над полем \( \mathrm F \), дополненное операцией \( L \times L \to L \), которая обозначается \( (x ,y) \mapsto [x, y] \) и называется \textit{скобкой Ли} или \textit{коммутатором} \( x \) и \( y \), называется \textit{алгеброй Ли} над полем \( \mathrm F \), если выполнен следующий ряд аксиом:
\begin{itemize}
    \item[(\( L 1 \))] Скобка Ли билинейна.
    \item[(\( L 2 \))] \( [x, x] = 0 \) для любого \( x \in L \).
    \item[(\( L 3 \))] \( [x, [y, z]] + [y, [z, x]] + [z, [x, y]] = 0 \quad (x, y, z \in L) \).
\end{itemize}
\end{definition}

Аксиома (\( L 3 \)) называется \textit{тождеством Якоби}.
Из аксиом (\( L 1 \)) и (\( L 2 \)), применённых к скобке \( [x + y, x + y] \), следует антикоммутативность скобки Ли:
\begin{itemize}
    \item[(\( L 2' \))] \( [x, y] = -[y, x] \) для любых \( x, y \in L \).
\end{itemize}
Обратно, если \( \operatorname{char} \mathrm F \neq 2 \), то из утверждение (\( L 2' \)) тривиально следует из аксиомы (\( L 2 \)) и потому для таких полей (\( L 2 \)) эквивалентна (\( L 2' \)).

\begin{definition}
    \label{def:isomorphous_algebras}
    Две алгебры Ли \( L \), \( L' \) называются \textit{изоморфными}, если существует такой изоморфизм векторных пространств \( \phi : L \to L' \), что
    \[
        \phi([x, y]) = [\phi(x), \phi(y)] \quad \forall x, y \in L.
    \]
    Само отображение \( \phi \) при этом называется \textit{изоморфизмом} алгебр Ли.
\end{definition}

\begin{definition}
    Подпространство \( K \) алгебры Ли \( L \) называется \textit{подалгеброй} алгебры \( L \), если \( K \) замкнуто относительно скобки Ли, т.е.
    \[
        \forall x, y \in K \quad [x, y] \in K.
    \]
\end{definition}

Нетрудно показать, что само подпространство \( K \) вместе с индуцированными операциями также является алгеброй Ли.

Любая алгебра Ли \( L \) имеет как минимум две тривиальные (\textit{несобственные}) подалгебры отвечающие тривиальным подпространствам: \( \left\{ 0 \right\} \) и \( L \).
Также, если \( L \neq \left\{ 0 \right\} \), то любой ненулевой элемент \( v \in L \) задаёт одномерную подалгебру \( Fv \). Умножение в такой алгебре тривиально, поскольку в силу аксиом (\( L 1 \)) и (\( L 2 \)) имеем \( [x, y] = 0 \) для любых \( x, y \in Fv \).

\subsection{Линейные алгебры Ли}
Пусть \( V \) --- конечномерное векторное пространство на полем \( \mathrm F \). Обозначим через \( \operatorname{End} V \) множество всех эндоморфизмов в пространстве \( V \). Тогда \( \operatorname{End} V \) --- векторное пространство размерности \( n^2 \) (где \( n = \dim V \)) над полем \( \mathrm F \) и одновременно \( \operatorname{End} V \) --- кольцо относительно обычной операции умножения. Определим новую операцию \( [x, y] = xy - yx \), называемую \textit{скобкой} или \textit{коммутатором} элементов \( x \) и \( y \). Вместе с ней \( \operatorname{End} V \) становится алгеброй Ли над полем \( \mathrm F \): выполнение аксиом (\( L 1 \)) и (\( L 2 \)) очевидно, а аксиома (\( L 3 \)) напрямую следует из (\( L 1 \)) и (\( L 2 \)). Чтобы отличать полученную алгебраическую структуру от изначальной ассоциативной структуры кольца, мы будем обозначать \( \operatorname{End} V \) как \( \mathfrak{gl}(V) \), когда она рассматривается как алгебра Ли.

\begin{definition}
    Алгебра \( \mathfrak{gl}(V) \) называется \textit{полной линейной алгеброй}.
\end{definition}

\begin{definition}
    Любая подалгебра \( \mathfrak{gl}(V) \) называется \textit{линейной алгеброй Ли}.
\end{definition}

Зафиксировав базис в пространстве \( V \), можно отождествить \( \mathfrak{gl}(V) \) с множеством всех матриц размера \( n \times n \) над полем \( \mathrm F \), обозначаемым \( \mathfrak{gl}(n, \mathrm F) \), что удобно при выполнении вычислений в явном виде.

Рассмотрим теперь некоторые другие примеры, играющие основную роль в этой работе наряду с \( \mathfrak{gl}(V) \). Они распадаются на четыре семейства: \( \mathbf{A}_l, \mathbf{B}_l, \mathbf{C}_l, \mathbf{D}_l \) (где \( l \geqslant 1 \)) --- и называются \textit{классическими алгебрами Ли}. В примерах \( \mathbf{B}_l \) ---  \( \mathbf{D}_l \) будем считать, что \( \operatorname{char} \mathrm F \neq 2 \).

\paragraph{\( \mathbf{A}_l \):}
Пусть \( \dim V = l + 1 \). Обозначим через \( \mathfrak{sl}(V) \) или \( \mathfrak{sl}(l + 1, \mathrm F) \) множество всех эндоморфизмов в пространстве \( V \), имеющих нулевой след. Поскольку
\begin{align*}
    \operatorname{Tr}(xy) &= \operatorname{Tr}(yx), \\
    \operatorname{Tr}(x + y) &= \operatorname{Tr}(x) + \operatorname{Tr}(y),
\end{align*}
множество \( \mathfrak{sl}(V) \) замкнуто относительно коммутирования и потому является подалгеброй \( \mathfrak{gl}(V) \), называемой \textit{специальной линейной алгеброй}.

Найдём теперь размерность \( \mathfrak{sl}(V) \). С~одной стороны \( \mathfrak{sl}(V) \) --- собственная подалгебра \( \mathfrak{gl}(V) \), так что её размерность не может быть больше \({ (l + 1)^2 - 1 }\). С~другой стороны нетрудно явно предоставить такое количество линейно независимых матриц с нулевым следом:
\[
    \left\{ e_{ij} \mid i \neq j \right\} \cup \left\{ e_{ii} - e_{i + 1, i + 1} \mid 1 \leqslant i \leqslant l \right\},
\]
где \( e_{ij} \) --- матрица у которой в позиции \({ (i, j) }\) стоит единица, а в остальных позициях --- ноль. Этот базис будем считать стандартным в пространстве \({ \mathfrak{sl}(l + 1, \mathrm F) }\).

\paragraph{\( \mathbf{C}_l \):} \ldots
\paragraph{\( \mathbf{B}_l \):} \ldots
\paragraph{\( \mathbf{D}_l \):} \ldots \\

Отметим также несколько примеров, играющих далее вспомогательную роль. Пусть \( \mathfrak{t}(n, \mathrm F) \) --- множество всех верхнетреугольных матриц, \( \mathfrak{n}(n, \mathrm F) \) --- множество строго верхнетреугольных матриц и \( \mathfrak{d}(n, \mathrm F) \) --- множество всех диагональных матриц. Тривиально проверяется, что каждое из этих множеств замкнуто относительно коммутирования и потому является линейной алгеброй.

\end{document}

% vim: spell spelllang=ru_yo
