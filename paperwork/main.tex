\documentclass[a4paper, 12pt]{article}
\usepackage[a4paper]{geometry}

\usepackage[T1,T2A]{fontenc}
\usepackage[utf8]{inputenc}
\usepackage[english,russian]{babel}
\usepackage{libertine}

\usepackage{amsmath}
\usepackage{amssymb}
\usepackage{amsthm}
\usepackage{mathrsfs}
\usepackage{mathtools}

\title{Нильпотентные и разрешимые алгебры Ли}
\author{Виногродский Серафим}
\date{\today}

\newtheorem{theorem}{Теорема}[section]
\newtheorem{corollary}{Следствие}[theorem]
\newtheorem{lemma}[theorem]{Лемма}

\theoremstyle{definition}
\newtheorem{definition}{Определение}[section]

\begin{document}
\maketitle
\tableofcontents
\thispagestyle{empty}
\pagebreak

\section{Введение}%
\label{sec:introduction}

\subsection{Основные понятия}%
\label{sec:lie_algebra_notion}

\begin{definition}
    \label{def:lie_algebra}
    Векторное пространство \( L \) над полем \( \mathrm F \), дополненное операцией \( L \times L \to L \), которая обозначается \( (x ,y) \mapsto [x, y] \) и называется \textit{скобкой Ли} или \textit{коммутатором} \( x \) и \( y \), называется \textit{алгеброй Ли} над полем \( \mathrm F \), если выполнен следующий ряд аксиом:
\begin{itemize}
    \item[(\( L 1 \))] Скобка Ли билинейна.
    \item[(\( L 2 \))] \( [x, x] = 0 \) для любого \( x \in L \).
    \item[(\( L 3 \))] \( [x, [y, z]] + [y, [z, x]] + [z, [x, y]] = 0 \quad (x, y, z \in L) \).
\end{itemize}
\end{definition}

Аксиома (\( L 3 \)) называется \textit{тождеством Якоби}.
Из аксиом (\( L 1 \)) и (\( L 2 \)), применённых к скобке \( [x + y, x + y] \), следует антикоммутативность скобки Ли:
\begin{itemize}
    \item[(\( L 2' \))] \( [x, y] = -[y, x] \) для любых \( x, y \in L \).
\end{itemize}
Обратно, если \( \operatorname{char} \mathrm F \neq 2 \), то из (\( L 2' \)) тривиально следует (\( L 2 \)) и потому для таких полей (\( L 2 \)) эквивалентна (\( L 2' \)).

\begin{definition}
    \label{def:isomorphous_algebras}
    Две алгебры Ли \( L \), \( L' \) называются \textit{изоморфными}, если существует такой изоморфизм векторных пространств \( \phi : L \to L' \), что
    \[
        \phi([x, y]) = [\phi(x), \phi(y)] \quad \forall x, y \in L.
    \]
    Само отображение \( \phi \) при этом называется \textit{изоморфизмом} алгебр Ли.
\end{definition}

\begin{definition}
    Подпространство \( K \) алгебры Ли \( L \) называется \textit{подалгеброй} алгебры \( L \), если \( K \) замкнуто относительно скобки Ли, т.е.
    \[
        \forall x, y \in K \quad [x, y] \in K.
    \]
\end{definition}

Нетрудно показать, что само подпространство \( K \) вместе с индуцированными операциями также является алгеброй Ли.

Любая алгебра Ли \( L \) имеет как минимум две тривиальные (\textit{несобственные}) подалгебры отвечающие тривиальным подпространствам: \( \left\{ 0 \right\} \) и \( L \).
Помимо этого любой ненулевой элемент \( v \in L \), если таковой имеется, определяет одномерную подалгебру \( Fv \) с тривиальным умножением, поскольку в силу (\( L 1 \)) и (\( L 2 \)) имеем \( [x, y] = 0 \) для любых \( x, y \in Fv \).

\end{document}

% vim: spell spelllang=ru_yo
